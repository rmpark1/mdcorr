% Preamble inputs
\documentclass[11pt, a4paper]{article}
\input{~/setup/mac/config/latex/templates/preamble.tex}
\input{~/setup/mac/config/latex/templates/diff-macros.tex}
% \input{~/setup/mac/config/latex/templates/header.tex}
\input{~/setup/mac/config/latex/templates/vector-calc-macros.tex}

\title{MDCorr Notes}
\author{Ryan Park}
\date{\today}

\begin{document}
\input{~/setup/mac/config/latex/templates/length-adjust.tex}
\maketitle
\tableofcontents
\renewcommand{\vec}[1]{\boldsymbol{#1}}

\section{Correlations}

There are general relationships between correlations and Fourier transforms. Especially,
there is the convolution theorem. Define the convolution as
\begin{equation}\begin{aligned}
    g\p{t} * h\p{t} \equiv \int_{-\infty}^{\infty} g\p{\tau} h\p{t - \tau} \diff \tau. \\
\end{aligned}\end{equation}
Define the Fourier transform as
\begin{equation}\begin{aligned}
    H\p{f} \equiv \mathcal{F} \cu{h\p{t}} \p{f} = \int_{- \infty}^{\infty} h\p{t} e^{2 \pi i f t} \diff t
\end{aligned}\end{equation}
and the inverse,
\begin{equation}\begin{aligned}
    h\p{t} = \mathcal{F}^{-1}\cu{H\p{f}} = \int_{-\infty}^{\infty} H\p{f} e^{-2 \pi i f t} \diff f
\end{aligned}\end{equation}
The convolution theorem states:
\begin{equation}\begin{aligned}
    \mathcal{F} \cu{g * h} = G\p{f} H\p{f} \\
\end{aligned}\end{equation}
The correlation of two functions may be written as,
\begin{equation}\begin{aligned}
    \text{Corr}\p{g, h} = g\p{t} * h\p{-t} \\
\end{aligned}\end{equation}
If both $g$ and $h$ are real, then
\begin{equation}\begin{aligned}
    \text{Corr} \p{g, h} &= G\p{f} H\p{f}^{*} \\
    \text{Corr}\p{g, h} &= \mathcal{F}^{-1} \cu{ \mathcal{F} \cu{g} \mathcal{F}\cu{h}^{*}} \\
\end{aligned}\end{equation}
If the correlation is with the same real function~\cite{Press1992}, i.e., it is a real autocorrelation,
then the solution takes a simple form
\begin{equation}\begin{aligned}
    \text{Corr}\p{g, g} = \mathcal{F}^{-1} \cu{\abs{G}^{2}}.
\end{aligned}\end{equation}
This is known as the ``Wiener-Khinchin Theorem''.

\section{Discretization}
The Discrete Fourier Transform (DFT) of a complex series $x_{n}$ with size $N$ is computed as
\begin{equation}\begin{aligned}
    X_{k} = \sum_{n}^{N-1} x_{n} e^{2 \pi i k n / N}, \\
\end{aligned}\end{equation}
and its inverse,
\begin{equation}\begin{aligned}
    x_{n} = \frac{1}{N}\sum_{k=0}^{N-1} X_{k} e^{-2 \pi i k n /N}. \\
\end{aligned}\end{equation}
There is a corresponding discrete theorem called the circular convolution theorem. Because the sum is
finite, the Fourier decomposition is periodc in $k$, and the resulting correlations of the real space
array will include contributions between the beginning and end of the input array, as if it were
wrapped into a ring and convolved with itself. To exclude these contributions, i.e. to compute the linear
convolution as opposed to the circular one, the input array must be padded with zeros such that the
wrapping does not affect length scales longer than the original input array.

\section{FFT}
The DFT can be denoted,
\begin{equation}\begin{aligned}
    \vec{X} &= \mathcal{F}_{N}\cu{\vec{x}} \\
    \vec{x} &= \mathcal{F}_{N}^{-1}\cu{\vec{X}} \\
\end{aligned}\end{equation}
where $N$ is the size of the input and output vectors of the tranform operation.
Now suppose $N$ is divisible by some integer $a$.
The sum can be decomposed into parts,
\begin{equation}\begin{aligned}
    X_{k}
    &= \sum_{s=0}^{a-1} \sum_{m=0}^{ N/a } x_{am + s} e^{2 \pi i k \p{am + s} / N} \\
    &= \sum_{s=0}^{a-1} e^{2 \pi i k s/N}\sum_{m=0}^{ N/a } x_{am + s} e^{2 \pi i k \p{am} / N} \\
    &= \sum_{s=0}^{a-1} e^{2 \pi i k s/N}\sum_{m=0}^{ N/a } x_{am + s} e^{2 \pi i k m / \p{N/a}} \\
    \mathcal{F}_{N} \cu{\vec{x}}_{k} &= \sum_{s=0}^{a-1} e^{2 \pi i k s/N} \mathcal{F}_{N/a} \cu{\vec{x}}_{k} \\
\end{aligned}\end{equation}
Thus the Fourier transform can be decomposed into the sum of smaller Fourier transforms. This results in a computational
speedup because $\mathcal{F}_{N/a} \cu{\vec{x}}_{k}$ is periodic in $k$ with periodicity $N/a$. Thus
\begin{equation}\begin{aligned}
    \mathcal{F}_{N/a} \cu{\vec{x}}_{k} = \mathcal{F}_{N/a} \cu{\vec{x}}_{k \text{ mod } N/a}.
\end{aligned}\end{equation}
Therefore
\begin{equation}\begin{aligned}
    \mathcal{F}_{N} \cu{\vec{x}}_{k} &= \sum_{s=0}^{a-1} e^{2 \pi i k s/N} \mathcal{F}_{N/a} \cu{\vec{x}}_{k \text{ mod } N/a}. \\
\end{aligned}\end{equation}
Similarly,
\begin{equation}\begin{aligned}
    \mathcal{F}^{-1}_{N} \cu{\vec{x}}_{n} = \frac{1}{N} \sum_{s=0}^{a-1} e^{-2 \pi i n s /N} \mathcal{F}_{N/a}^{-1} \cu{\vec{X}}_{n \text{ mod } N/a}.
\end{aligned}\end{equation}
The single thread runtime of this substep $l$ with size $n_{l}$ is
\begin{equation}\begin{aligned}
    R_{l}\p{n_{l}} &= a_{l} n_{l} + a_{l}R_{l-1} \p{n_{l}/a_{l}} \\
    &= a_{l} \p{n_{l} + R_{l-1}\p{n_{l}/a_{l}}} \\
\end{aligned}\end{equation}
This is because each $k \in 0, ..., n_{s}-1$ must be evaluated. Assuming $a_{s}$ $R_{s+1}$ subproblems have already been performed,
that leaves an $O\p{1}$ read operation from the child calculation and $a$ butterfly operations for each $k$.
If $N$ has the prime factorization $a_{p}$ of length $P$, then there will be $P$ layers. Enumerate these layers starting from
a base case $a_{1}, a_{2}, ..., a_{P}$.
\begin{equation}\begin{aligned}
    R_{2} &= a_{2} \p{n_{2} + R_{1}} \\
    R_{3} &= a_{3} \p{n_{3} + a_{2} \p{n_{2} + R_{1}}} \\
    R_{4} &= a_{4} \p{n_{4} + a_{3} \p{n_{3} + a_{2} \p{n_{2} + R_{1}}}} \\
          &= a_{4} n_{4} + a_{4} a_{3} n_{3} + a_{4} a_{3} a_{2} n_{2} + a_{4} a_{3} a_{2} R_{1} \\
          &\vdots \\
    R_{P} =  R \p{N} &= \sum_{l=2}^{P} n_{l} \prod_{p=2}^{P} a_{p} + R_{1} \prod_{p=1}^{P} a_{p} \\
    &=  \sum_{l=2}^{P} n_{l} \prod_{p=l}^{P} a_{p} + R_{1} N \\
\end{aligned}\end{equation}
The size of each layer is given by
\begin{equation}\begin{aligned}
    n_{l} = \prod_{p=1}^{l-1} a_{p}
\end{aligned}\end{equation}
Thus
\begin{equation}\begin{aligned}
    R = \sum_{l=1}^{P} a_{l} \prod_{p=1}^{P} a_{p} = N \sum_{l=1}^{P} a_{l}
\end{aligned}\end{equation}
% kThe runtime is clearly optimized whe
The indices respect a similar recursive relationship. Adding level indices to indicate
location of data,
\begin{equation}\begin{aligned}
    \mathcal{F}_{s,n_{l}} 
\end{aligned}\end{equation}
\begin{equation}\begin{aligned}
    x_{n,l} = a_{l} x_{n,l} + s_{l}
\end{aligned}\end{equation}
In explicit form,
\begin{equation}\begin{aligned}
    x_{n,l-1} &= a_{l} x_{n, l} + s_{l} \\
    x_{n+1, l} &= \frac{x_{n_{l}} - s_{l}}{a_{l}} \\
\end{aligned}\end{equation}
The remapping for the last layer index should be
\begin{equation}\begin{aligned}
    x_{j}^{\abs{a}} \leftarrow N \p{\cancel{x_{i}}^{0}} + \sum_{p=\abs{a}-1}^{0} \p{\prod_{r=0}^{p-1} a_{r}} s_{p}\p{j}
\end{aligned}\end{equation}
But
\begin{equation}\begin{aligned}
    s_{p}\p{j} = j \mod \prod_{r=0}^{p-1} a_{r}
\end{aligned}\end{equation}

\begin{itemize}
    \item explicit index mapping
    \item complex number respresentation.
\end{itemize}

\begin{itemize}
    \item The RDF cutoff must coordinate with the ghost mapping when the realspace cutoff is small.
    \item This can be fixed with the command
    \begin{lstlisting}
        comm_modify cutoff {RDF cutoff + skin}.
    \end{lstlisting}
\end{itemize}


\bibliographystyle{plain}
\bibliography{refs}

\end{document}
